\nonstopmode{}
\documentclass[letterpaper]{book}
\usepackage[times,inconsolata,hyper]{Rd}
\usepackage{makeidx}
\usepackage[utf8]{inputenc} % @SET ENCODING@
% \usepackage{graphicx} % @USE GRAPHICX@
\makeindex{}
\begin{document}
\chapter*{}
\begin{center}
{\textbf{\huge Package `zipfextR'}}
\par\bigskip{\large \today}
\end{center}
\begin{description}
\raggedright{}
\inputencoding{utf8}
\item[Type]\AsIs{Package}
\item[Title]\AsIs{Zipf Extended Distributions}
\item[Version]\AsIs{0.6.0}
\item[Author]\AsIs{Ariel Duarte-López and Marta Pérez-Casany}
\item[Maintainer]\AsIs{Ariel Duarte-López }\email{aduarte@ac.upc.edu}\AsIs{}
\item[Description]\AsIs{Implementation of three extensions of the Zipf distribution: the Marshall-Olkin
Extended Zipf (MOEZipf), the Zipf-Poisson Extreme (Zipf-PE) and the
Zipf-Poisson Stopped Sum (Zipf-PSS) distributions. The two first extensions allow for
top-concavity (top-convexity) while the third one only allows for concavity.
In log-log scale, all the extensions maintain the linearity, associated with the
Zipf model, in the tail. This can be appreciated by plotting the probabilities in
log-log scale.}
\item[License]\AsIs{GPL-3}
\item[Depends]\AsIs{R (>= 2.0.1)}
\item[Imports]\AsIs{VGAM (>= 0.9.8), tolerance(>= 1.2.0)}
\item[Encoding]\AsIs{UTF-8}
\item[LazyData]\AsIs{true}
\item[URL]\AsIs{}\url{https://github.com/ardlop/zipfextR}\AsIs{}
\item[BugReports]\AsIs{}\url{https://github.com/ardlop/zipfextR/issues}\AsIs{}
\item[RoxygenNote]\AsIs{6.0.1}
\item[Suggests]\AsIs{testthat}
\end{description}
\Rdcontents{\R{} topics documented:}
\inputencoding{utf8}
\HeaderA{moezipf}{The Marshal-Olkin Extended Zipf Distribution (MOEZipf).}{moezipf}
\aliasA{dmoezipf}{moezipf}{dmoezipf}
\aliasA{dmoezipf}{moezipf}{dmoezipf}
\aliasA{pmoezipf}{moezipf}{pmoezipf}
\aliasA{qmoezipf}{moezipf}{qmoezipf}
\aliasA{rmoezipf}{moezipf}{rmoezipf}
%
\begin{Description}\relax
Probability mass function, cumulative function, quantile function and random number
generation for the MOEZipf distribution with parameters \eqn{\alpha}{} and \eqn{\beta}{}.
\end{Description}
%
\begin{Usage}
\begin{verbatim}
dmoezipf(x, alpha, beta, log = FALSE)

pmoezipf(q, alpha, beta, log.p = FALSE, lower.tail = TRUE)

qmoezipf(p, alpha, beta, log.p = FALSE, lower.tail = TRUE)

rmoezipf(n, alpha, beta)
\end{verbatim}
\end{Usage}
%
\begin{Arguments}
\begin{ldescription}
\item[\code{x, q}] Vector of positive integer values.

\item[\code{alpha}] Value of the \eqn{\alpha}{} parameter (\eqn{\alpha > 1}{} ).

\item[\code{beta}] Value of the \eqn{\beta}{} parameter (\eqn{\beta > 0}{} ).

\item[\code{log, log.p}] Logical; if TRUE, probabilities p are given as log(p).

\item[\code{lower.tail}] Logical; if TRUE (default), probabilities are \eqn{P[X \leq x]}{}, otherwise, \eqn{P[X > x]}{}.

\item[\code{p}] Vector of probabilities.

\item[\code{n}] Number of random values to return.
\end{ldescription}
\end{Arguments}
%
\begin{Details}\relax
The \emph{probability mass function} at a positive integer value \eqn{x}{} of the MOEZipf distribution with
parameters \eqn{\alpha}{} and \eqn{\beta}{} is computed as follows:

\deqn{p(x | \alpha, \beta) = \frac{x^{-\alpha} \beta \zeta(\alpha) }{[\zeta(\alpha) - \bar{\beta} \zeta (\alpha, x)] [\zeta (\alpha) - \bar{\beta} \zeta (\alpha, x + 1)]},\, x = 1,2,...,\, \alpha > 1, \beta > 0, }{}

where \eqn{\zeta(\alpha)}{} is the Riemann-zeta function at \eqn{\alpha}{}, \eqn{\zeta(\alpha, x)}{}
is the Hurtwitz zeta function with arguments \eqn{\alpha}{} and x, and \eqn{\bar{\beta} = 1 - \beta}{}.

The \emph{cumulative distribution function}, at a given positive integer value \eqn{x}{},
is computed as \eqn{F(x) = 1 - S(x)}{}, where the survival function \eqn{S(x)}{} is equal to:
\deqn{S(x) = \frac{\beta\, \zeta(\alpha, x + 1)}{\zeta(\alpha) - \bar{\beta}\,\zeta(\alpha, x + 1)},\, x = 1, 2, .. }{}

The quantile of the MOEZipf\eqn{(\alpha, \beta)}{} distribution of a given probability value p,
is equal to the quantile of the Zipf\eqn{(\alpha)}{} distribution at the value:
\deqn{p\prime = \frac{p\,\beta}{1 + p\,(\beta - 1)},\, \, \, (1)}{}

The quantiles of the Zipf\eqn{(\alpha)}{} distribution are computed by means of the \emph{tolerance}
package.

The random generator function applies the \emph{quantile} function over \emph{n} values
from an Uniform distribution in the interval (0, 1,) in order to obtain
the random values.
\end{Details}
%
\begin{Value}

\code{dmoezipf} gives the probability mass function,
\code{pmoezipf} gives the cumulative distribution function,
\code{qmoezipf} gives the quantile function, and
\code{rmoezipf} generates random values from a MOEZipf distribution.
\end{Value}
%
\begin{References}\relax

Casellas, A. (2013) \emph{La distribució Zipf Estesa segons la transformació Marshall-Olkin}. Universitat Politécnica de Catalunya.

Devroye L. (1986) Non-Uniform Random Variate Generation. Springer, New York, NY.

Duarte-López, A., Prat-Pérez, A., \& Pérez-Casany, M. (2015, August). \emph{Using the Marshall-Olkin Extended Zipf Distribution in Graph Generation}. In European Conference on Parallel Processing (pp. 493-502). Springer International Publishing.

Pérez-Casany, M. and Casellas, A. (2013) \emph{Marshall-Olkin Extended Zipf Distribution}. arXiv preprint arXiv:1304.4540.

Young, D. S. (2010). \emph{Tolerance: an R package for estimating tolerance intervals}. Journal of Statistical Software, 36(5), 1-39.

\end{References}
%
\begin{Examples}
\begin{ExampleCode}
dmoezipf(1:10, 2.5, 1.3)
pmoezipf(1:10, 2.5, 1.3)
qmoezipf(0.56, 2.5, 1.3)
rmoezipf(10, 2.5, 1.3)

\end{ExampleCode}
\end{Examples}
\inputencoding{utf8}
\HeaderA{moezipfFit}{MOEZipf parameters estimation.}{moezipfFit}
\aliasA{AIC.moezipfR}{moezipfFit}{AIC.moezipfR}
\aliasA{BIC.moezipfR}{moezipfFit}{BIC.moezipfR}
\aliasA{coef.moezipfR}{moezipfFit}{coef.moezipfR}
\aliasA{fitted.moezipfR}{moezipfFit}{fitted.moezipfR}
\aliasA{logLik.moezipfR}{moezipfFit}{logLik.moezipfR}
\aliasA{plot.moezipfR}{moezipfFit}{plot.moezipfR}
\aliasA{print.moezipfR}{moezipfFit}{print.moezipfR}
\aliasA{residuals.moezipfR}{moezipfFit}{residuals.moezipfR}
\aliasA{summary.moezipfR}{moezipfFit}{summary.moezipfR}
%
\begin{Description}\relax
For a given sample of strictly positive integer numbers,  usually of the type of ranking data or
frequencies of frequencies data, estimates the parameters of a MOEZipf distribution by means of
the maximum likelihood method. Note that the input data should be provided as a frequency matrix.
\end{Description}
%
\begin{Usage}
\begin{verbatim}
moezipfFit(data, init_alpha, init_beta, level = 0.95, ...)

## S3 method for class 'moezipfR'
residuals(object, ...)

## S3 method for class 'moezipfR'
fitted(object, ...)

## S3 method for class 'moezipfR'
coef(object, ...)

## S3 method for class 'moezipfR'
plot(x, ...)

## S3 method for class 'moezipfR'
print(x, ...)

## S3 method for class 'moezipfR'
summary(object, ...)

## S3 method for class 'moezipfR'
logLik(object, ...)

## S3 method for class 'moezipfR'
AIC(object, ...)

## S3 method for class 'moezipfR'
BIC(object, ...)
\end{verbatim}
\end{Usage}
%
\begin{Arguments}
\begin{ldescription}
\item[\code{data}] Matrix of count data in form of table of frequencies.

\item[\code{init\_alpha}] Initial value of \eqn{\alpha}{} parameter (\eqn{\alpha > 1}{}).

\item[\code{init\_beta}] Initial value of \eqn{\beta}{} parameter (\eqn{\beta > 0}{}).

\item[\code{level}] Confidence level used to calculate the confidence intervals (default 0.95).

\item[\code{...}] Further arguments to the generic functions. The extra arguments are passing to the \emph{\LinkA{optim}{optim}} function.

\item[\code{object}] An object from class "moezipfR" (output of \emph{moezipfFit} function).

\item[\code{x}] An object from class "moezipfR" (output of \emph{moezipfFit} function).
\end{ldescription}
\end{Arguments}
%
\begin{Details}\relax
The argument \code{data} is a two column matrix such that the first column of each row contains a
count, while the corresponding second column contains its frequency.

The log-likelihood function is equalt to:

\deqn{l(\alpha, \beta; x) = -\alpha \sum_{i = 1} ^m f_{a}(x_{i}) log(x_{i}) + N (log(\beta) + \log(\zeta(\alpha)))}{}
\deqn{ - \sum_{i = 1} ^m f_a(x_i) log[(\zeta(\alpha) - \bar{\beta}\zeta(\alpha, x_i)(\zeta(\alpha) - \bar{\beta}\zeta(\alpha, x_i + 1)))], }{}
where \eqn{m}{} is the number of different values in the sample, \eqn{N}{} is the sample size,
i.e.  \eqn{N = \sum_{i = 1} ^m x_i f_a(x_i)}{},  being \eqn{f_{a}(x_i)}{} is the absolute
frequency of \eqn{x_i}{}.

The function \emph{\LinkA{optim}{optim}} is used to estimate the parameters.
\end{Details}
%
\begin{Value}
Returns a \emph{moezipfR} object composed by the maximum likelihood parameter estimations,
their standard deviation, their confidence intervals and the value of the log-likelihood at the
maximum likelihood estimator.
\end{Value}
%
\begin{SeeAlso}\relax
\code{\LinkA{moezipf\_getInitialValues}{moezipf.Rul.getInitialValues}}.
\end{SeeAlso}
%
\begin{Examples}
\begin{ExampleCode}
data <- rmoezipf(100, 2.5, 1.3)
data <- as.data.frame(table(data))
data[,1] <- as.numeric(data[,1])
initValues <- moezipf_getInitialValues(data)
obj <- moezipfFit(data, init_alpha = initValues$init_alpha, init_beta = initValues$init_beta)
\end{ExampleCode}
\end{Examples}
\inputencoding{utf8}
\HeaderA{moezipfMean}{Expected value.}{moezipfMean}
%
\begin{Description}\relax
Computes the expected value of the MOEZipf distribution for given values of parameters
\eqn{\alpha}{} and \eqn{\beta}{}.
\end{Description}
%
\begin{Usage}
\begin{verbatim}
moezipfMean(alpha, beta, tolerance = 10^(-4))
\end{verbatim}
\end{Usage}
%
\begin{Arguments}
\begin{ldescription}
\item[\code{alpha}] Value of the \eqn{\alpha}{} parameter (\eqn{\alpha > 2}{}).

\item[\code{beta}] Value of the \eqn{\beta}{} parameter (\eqn{\beta > 0}{}).

\item[\code{tolerance}] Tolerance used in the calculations (default = \eqn{10^{-4}}{}).
\end{ldescription}
\end{Arguments}
%
\begin{Details}\relax
The expected value of the MOEZipf distribution only exists for \eqn{\alpha}{} values strictly greater than 2.
In this case, if Y is a random variable that follows a MOEZipf distribution with parameters \eqn{\alpha}{}
and \eqn{\beta}{}, the expected value is equal to:
\deqn{E(Y) = \sum_{x = 1} ^\infty \frac{\beta \zeta(\alpha) x^{-\alpha + 1}}{[\zeta(\alpha) - \bar{\beta}\zeta(\alpha, x)][\zeta(\alpha) - \bar{\beta}\zeta(\alpha, x + 1)]},\, \alpha > 2,\,\beta > 0}{}

The mean is computed calculating the partial sums of the serie, and stopping when two
consecutive partial sums differs less than the \code{tolerance} value.
The value of the last partial sum is returned.
\end{Details}
%
\begin{Value}
A positive real value corresponding to the mean value of the distribution.
\end{Value}
%
\begin{Examples}
\begin{ExampleCode}
moezipfMean(2.5, 1.3)
moezipfMean(2.5, 1.3, 10^(-3))
\end{ExampleCode}
\end{Examples}
\inputencoding{utf8}
\HeaderA{moezipfMoments}{Distribution Moments.}{moezipfMoments}
%
\begin{Description}\relax
General function to compute the k-th moment of the MOEZipf distribution, for any integer value \eqn{k \geq 1}{}
when it exists. Note that the k-th moment exists if and only if  \eqn{\alpha > k + 1}{}.
When k = 1, this function returns the same value as the
\LinkA{moezipfMean}{moezipfMean} function.
\end{Description}
%
\begin{Usage}
\begin{verbatim}
moezipfMoments(k, alpha, beta, tolerance = 10^(-4))
\end{verbatim}
\end{Usage}
%
\begin{Arguments}
\begin{ldescription}
\item[\code{k}] Order of the moment to compute.

\item[\code{alpha}] Value of the \eqn{\alpha}{} parameter (\eqn{\alpha > k + 1}{}).

\item[\code{beta}] Value of the \eqn{\beta}{} parameter (\eqn{\beta > 0}{}).

\item[\code{tolerance}] Tolerance used in the calculations (default = \eqn{10^{-4}}{}).
\end{ldescription}
\end{Arguments}
%
\begin{Details}\relax
The k-th moment of the MOEZipf distribution is finite for \eqn{\alpha}{} values strictly greater than \eqn{k + 1}{}.
For a random variable Y that follows a MOEZipf distribution with parameters \eqn{\alpha}{} and \eqn{\beta}{},
the k-th moment is equal to:

\deqn{E(Y^k) = \sum_{x = 1} ^\infty \frac{\beta \zeta(\alpha) x^{-\alpha + k}}{[\zeta(\alpha) - \bar{\beta}\zeta(\alpha, x)][\zeta(\alpha) - \bar{\beta}\zeta(\alpha, x + 1)]}\,, \alpha \geq k + 1\,, \beta > 0}{}

The k-th moment is computed calculating the partial sums of the serie, and stopping when two
consecutive partial sums differs less than the \code{tolerance} value.
The value of the last partial sum is returned.
\end{Details}
%
\begin{Value}
A positive real value corresponding to the k-th moment of the distribution.
\end{Value}
%
\begin{Examples}
\begin{ExampleCode}
moezipfMoments(3, 4.5, 1.3)
moezipfMoments(3, 4.5, 1.3,  1*10^(-3))
\end{ExampleCode}
\end{Examples}
\inputencoding{utf8}
\HeaderA{moezipfVariance}{Variance of the MOEZipf distribution.}{moezipfVariance}
%
\begin{Description}\relax
Computes the variance of the MOEZipf distribution for given values of \eqn{\alpha}{} and \eqn{\beta}{}.
\end{Description}
%
\begin{Usage}
\begin{verbatim}
moezipfVariance(alpha, beta, tolerance = 10^(-4))
\end{verbatim}
\end{Usage}
%
\begin{Arguments}
\begin{ldescription}
\item[\code{alpha}] Value of the \eqn{\alpha}{} parameter (\eqn{\alpha > 3}{}).

\item[\code{beta}] Value of the \eqn{\beta}{} parameter (\eqn{\beta > 0}{}).

\item[\code{tolerance}] Tolerance used in the calculations. (default = \eqn{10^{-4}}{})
\end{ldescription}
\end{Arguments}
%
\begin{Details}\relax
The variance of the distribution only exists for \eqn{\alpha}{} strictly greater than 3.
In this case, it is calculated as:
\deqn{Var[Y] = E[Y^2] - (E[Y])^2}{},
where the first and the second moments are computed using the \emph{moezipfMoments} function.
\end{Details}
%
\begin{Value}
A positive real value corresponding to the variance of the distribution.
\end{Value}
%
\begin{Examples}
\begin{ExampleCode}
moezipfVariance(3.5, 1.3)
\end{ExampleCode}
\end{Examples}
\inputencoding{utf8}
\HeaderA{moezipf\_getInitialValues}{Calculates initial values for the \eqn{\alpha}{} and \eqn{\beta}{} parameters.}{moezipf.Rul.getInitialValues}
%
\begin{Description}\relax
Initial values of the parameters useful to find the maximum likelihood estimators and required
in the \emph{moezipfFit} procedure. They may be computed using the empirical absolute frequencies
of values one and two. The selection of robust initial values allows to reduce the number
of iterations which in turn, reduces the computation time.
In the case where one of the two first positive integer values does not appear in the data
set, the default values are set to be equal to \eqn{\alpha}{} = 1.0001 and \eqn{\beta}{} = 0.0001.
\end{Description}
%
\begin{Usage}
\begin{verbatim}
moezipf_getInitialValues(data)
\end{verbatim}
\end{Usage}
%
\begin{Arguments}
\begin{ldescription}
\item[\code{data}] Matrix of count data.
\end{ldescription}
\end{Arguments}
%
\begin{Details}\relax
The argument \code{data} is a two column matrix such that the first column of each row contains a
count, while the corresponding second column contains its frequency.

To obtain the initial value of \eqn{\alpha}{} and \eqn{\beta}{}, it is assumed that
the data come from a Zipf(\eqn{\alpha}{}) distribution. The initial value for \eqn{\beta}{}
is set to be equal to one, and the inital value for \eqn{\alpha}{}, denoted by \eqn{\alpha_0}{}, is obtained
equating the ratio of the theoretical probabilities at one and two to the corresponding empirical
ratio. this gives:

\deqn{\alpha_0 = log_2 \big (\frac{f_a(1)}{f_a(2)} \big)}{}
where \eqn{f_1}{} and \eqn{f_2}{} are the absolute frequencies of one and two in the sample.
\end{Details}
%
\begin{Value}
Returns the initial value for parameters \eqn{\alpha}{} and \eqn{\beta}{}.
\end{Value}
%
\begin{References}\relax
 Güney, Y., Tuaç, Y., \& Arslan, O. (2016). Marshall–Olkin distribution: parameter estimation and
application to cancer data. Journal of Applied Statistics, 1-13.
\end{References}
%
\begin{Examples}
\begin{ExampleCode}
data <- rmoezipf(100, 2.5, 1.3)
data <- as.data.frame(table(data))
initials <- moezipf_getInitialValues(data)
\end{ExampleCode}
\end{Examples}
\inputencoding{utf8}
\HeaderA{zipfpe}{The Zipf-Poisson Extreme Distribution (Zipf-PE).}{zipfpe}
\aliasA{dzipfpe}{zipfpe}{dzipfpe}
\aliasA{dzipfpe}{zipfpe}{dzipfpe}
\aliasA{pzipfpe}{zipfpe}{pzipfpe}
\aliasA{qzipfpe}{zipfpe}{qzipfpe}
\aliasA{rzipfpe}{zipfpe}{rzipfpe}
%
\begin{Description}\relax
Probability mass function, cumulative function, quantile function and random number generation
for the Zipf-PE distribution with parameters \eqn{\alpha}{} and \eqn{\beta}{}.
\end{Description}
%
\begin{Usage}
\begin{verbatim}
dzipfpe(x, alpha, beta, log = FALSE)

pzipfpe(q, alpha, beta, log.p = FALSE, lower.tail = TRUE)

qzipfpe(p, alpha, beta, log.p = FALSE, lower.tail = TRUE)

rzipfpe(n, alpha, beta)
\end{verbatim}
\end{Usage}
%
\begin{Arguments}
\begin{ldescription}
\item[\code{x, q}] Vector of positive integer values.

\item[\code{alpha}] Value of the \eqn{\alpha}{} parameter (\eqn{\alpha > 1}{} ).

\item[\code{beta}] Value of the \eqn{\beta}{} parameter (\eqn{\beta > 0}{} ).

\item[\code{log, log.p}] Logical; if TRUE, probabilities p are given as log(p).

\item[\code{lower.tail}] Logical; if TRUE (default), probabilities are \eqn{P[X \leq x]}{}, otherwise, \eqn{P[X > x]}{}.

\item[\code{p}] Vector of probabilities.

\item[\code{n}] Number of random values to return.
\end{ldescription}
\end{Arguments}
%
\begin{Details}\relax
The \emph{probability mass function} of the Zipf-PE distribution at a positive integer
value \eqn{x}{} with parameters \eqn{\alpha}{} and \eqn{\beta}{} is computed as follows:

\deqn{p(x | \alpha, \beta) = \frac{e^{\beta (1 - \frac{\zeta(\alpha, x)}{\zeta(\alpha)})} (e^{\beta \frac{x^{-\alpha}}{\zeta(\alpha)}} - 1)}
{e^{\beta} - 1},\, x= 1,2,...,\, \alpha > 1,\, -\infty < \beta < +\infty,}{}

where \eqn{\zeta(\alpha)}{} is the Riemann-zeta function at \eqn{\alpha}{}, and \eqn{\zeta(\alpha, x)}{}
is the Hurtwitz zeta function with arguments \eqn{\alpha}{} and x.

The \emph{cumulative distribution function}, \eqn{F(x)}{}, at a given positive integer value \eqn{x}{},
is calcuted as:
\deqn{F(x) = \frac{e^{\beta (1 - \frac{\zeta(\alpha, x + 1)}{\zeta(\alpha)})} - 1}{e^{\beta} -1}}{}

The quantile of the Zipf-PE\eqn{(\alpha, \beta)}{} distribution of a given probability value p,
is equal to the quantile of the Zipf\eqn{(\alpha)}{} distribution at the value:

\deqn{p\prime = \frac{log(p\, (e^{\beta} - 1) + 1)}{\beta}}{}
The quantiles of the Zipf\eqn{(\alpha)}{} distribution are computed by means of the \emph{tolerance}
package.

The random generator function applies the \emph{quantile} function over \emph{n} values
from an Uniform distribution in the interval (0, 1,) in order to obtain
the random values.
\end{Details}
%
\begin{Value}

\code{dzipfpe} gives the probability mass function,
\code{pzipfpe} gives the cumulative function,
\code{qzipfpe} gives the quantile function, and
\code{rzipfpe} generates random values from a Zipf-PE distribution.  
\end{Value}
%
\begin{References}\relax

Young, D. S. (2010). \emph{Tolerance: an R package for estimating tolerance intervals}. Journal of Statistical Software, 36(5), 1-39.

\end{References}
%
\begin{Examples}
\begin{ExampleCode}
dzipfpe(1:10, 2.5, -1.5)
pzipfpe(1:10, 2.5, -1.5)
qzipfpe(0.56, 2.5, 1.3)
rzipfpe(10, 2.5, 1.3)

\end{ExampleCode}
\end{Examples}
\inputencoding{utf8}
\HeaderA{zipfpeFit}{Zipf-PE parameters estimation.}{zipfpeFit}
\aliasA{AIC.zipfpeR}{zipfpeFit}{AIC.zipfpeR}
\aliasA{BIC.zipfpeR}{zipfpeFit}{BIC.zipfpeR}
\aliasA{coef.zipfpeR}{zipfpeFit}{coef.zipfpeR}
\aliasA{fitted.zipfpeR}{zipfpeFit}{fitted.zipfpeR}
\aliasA{logLik.zipfpeR}{zipfpeFit}{logLik.zipfpeR}
\aliasA{plot.zipfpeR}{zipfpeFit}{plot.zipfpeR}
\aliasA{print.zipfpeR}{zipfpeFit}{print.zipfpeR}
\aliasA{residuals.zipfpeR}{zipfpeFit}{residuals.zipfpeR}
\aliasA{summary.zipfpeR}{zipfpeFit}{summary.zipfpeR}
%
\begin{Description}\relax
For a given sample of strictly positive integer values,  usually of the type of ranking data or
frequencies of frequencies data, estimates the parameters of a Zipf-PE
distribution by means of the maximum likelihood method.Note that the input data
should be provided as a frequency matrix.
\end{Description}
%
\begin{Usage}
\begin{verbatim}
zipfpeFit(data, init_alpha, init_beta, level = 0.95, ...)

## S3 method for class 'zipfpeR'
residuals(object, ...)

## S3 method for class 'zipfpeR'
fitted(object, ...)

## S3 method for class 'zipfpeR'
coef(object, ...)

## S3 method for class 'zipfpeR'
plot(x, ...)

## S3 method for class 'zipfpeR'
print(x, ...)

## S3 method for class 'zipfpeR'
summary(object, ...)

## S3 method for class 'zipfpeR'
logLik(object, ...)

## S3 method for class 'zipfpeR'
AIC(object, ...)

## S3 method for class 'zipfpeR'
BIC(object, ...)
\end{verbatim}
\end{Usage}
%
\begin{Arguments}
\begin{ldescription}
\item[\code{data}] Matrix of count data in form of table of frequencies.

\item[\code{init\_alpha}] Initial value of \eqn{\alpha}{} parameter (\eqn{\alpha > 1}{}).

\item[\code{init\_beta}] Initial value of \eqn{\beta}{} parameter (\eqn{\beta \in (-\infty, +\infty)}{}).

\item[\code{level}] Confidence level used to calculate the confidence intervals (default 0.95).

\item[\code{...}] Further arguments to the generic functions.The extra arguments are passing
to the \emph{\LinkA{optim}{optim}} function.

\item[\code{object}] An object from class "zpeR" (output of \emph{zipfpeFit} function).

\item[\code{x}] An object from class "zpeR" (output of \emph{zipfpeFit} function).
\end{ldescription}
\end{Arguments}
%
\begin{Details}\relax
The argument \code{data} is a two column matrix such that the first column of each row contains a
count, while the corresponding second column contains its frequency.

The log-likelihood function is equal to:

\deqn{l(\alpha, \beta; x) = \beta\, (N - \zeta(\alpha)^{-1}\, \sum_{i = 1} ^m  f_{a}(x_{i})\, \zeta(\alpha, x_i)) +
\sum_{i = 1} ^m f_{a}(x_{i})\,  log \left( \frac{e^{\frac{\beta\, x_{i}^{-\alpha}}{\zeta(\alpha)}} - 1}{e^{\beta} - 1} \right), }{}
where \eqn{m}{} is the number of different values in the sample, \eqn{N}{} is the sample size,
i.e.  \eqn{N = \sum_{i = 1} ^m x_i f_a(x_i)}{},  being \eqn{f_{a}(x_i)}{} is the absolute
frequency of \eqn{x_i}{}.

The function \emph{\LinkA{optim}{optim}} is used to estimate the parameters.
\end{Details}
%
\begin{Value}
Returns an object composed by the maximum likelihood parameter estimations, their standard deviation, their confidence
intervals and the value of the log-likelihood at the maximum likelihood estimations.
\end{Value}
%
\begin{SeeAlso}\relax
\code{\LinkA{moezipf\_getInitialValues}{moezipf.Rul.getInitialValues}}.
\end{SeeAlso}
%
\begin{Examples}
\begin{ExampleCode}
data <- rzipfpe(100, 2.5, 1.3)
data <- as.data.frame(table(data))
data[,1] <- as.numeric(data[,1])
obj <- zipfpeFit(data, 1.1, 0.1)
\end{ExampleCode}
\end{Examples}
\inputencoding{utf8}
\HeaderA{zipfpeMean}{Expected value of the Zipf-PE distribution.}{zipfpeMean}
%
\begin{Description}\relax
Computes the expected value of the Zipf-PE distribution for given values of parameters
\eqn{\alpha}{} and \eqn{\beta}{}.
\end{Description}
%
\begin{Usage}
\begin{verbatim}
zipfpeMean(alpha, beta, tolerance = 10^(-4))
\end{verbatim}
\end{Usage}
%
\begin{Arguments}
\begin{ldescription}
\item[\code{alpha}] Value of the \eqn{\alpha}{} parameter (\eqn{\alpha > 2}{}).

\item[\code{beta}] Value of the \eqn{\beta}{} parameter (\eqn{\beta \in (-\infty, +\infty)}{}).

\item[\code{tolerance}] Tolerance used in the calculations (default = \eqn{10^{-4}}{}).
\end{ldescription}
\end{Arguments}
%
\begin{Details}\relax
The mean of the distribution only exists for \eqn{\alpha}{} strictly greater than 2.
It is computed calculating the partial sums of the serie, and stopping when two
consecutive partial sums differs less than the \code{tolerance} value.
The value of the last partial sum is returned.
\end{Details}
%
\begin{Value}
A positive real value corresponding to the mean value of the Zipf-PE distribution.
\end{Value}
%
\begin{Examples}
\begin{ExampleCode}
zipfpeMean(2.5, 1.3)
zipfpeMean(2.5, 1.3, 10^(-3))
\end{ExampleCode}
\end{Examples}
\inputencoding{utf8}
\HeaderA{zipfpeMoments}{Distribution Moments.}{zipfpeMoments}
%
\begin{Description}\relax
General function to compute the k-th moment of the Zipf-PE distribution, for any integer value
\eqn{k \geq 1}{} when it exists. Note that the k-th moment exists if and
only if \eqn{\alpha > k + 1}{}.
\end{Description}
%
\begin{Usage}
\begin{verbatim}
zipfpeMoments(k, alpha, beta, tolerance = 10^(-4))
\end{verbatim}
\end{Usage}
%
\begin{Arguments}
\begin{ldescription}
\item[\code{k}] Order of the moment to compute.

\item[\code{alpha}] Value of the \eqn{\alpha}{} parameter (\eqn{\alpha > k + 1}{}).

\item[\code{beta}] Value of the \eqn{\beta}{} parameter (\eqn{\beta \in (-\infty, +\infty)}{}).

\item[\code{tolerance}] Tolerance used in the calculations (default = \eqn{10^{-4}}{}).
\end{ldescription}
\end{Arguments}
%
\begin{Details}\relax
The k-th moment of the Zipf-PE distribution is finite for \eqn{\alpha}{} values strictly greater than \eqn{k + 1}{}.
It is computed calculating the partial sums of the serie, and stopping when two
consecutive partial sums differs less than the \code{tolerance} value.
The value of the last partial sum is returned.
\end{Details}
%
\begin{Value}
A positive real value corresponding to the k-th moment of the distribution.
\end{Value}
%
\begin{Examples}
\begin{ExampleCode}
zipfpeMoments(3, 4.5, 1.3)
zipfpeMoments(3, 4.5, 1.3,  1*10^(-3))
\end{ExampleCode}
\end{Examples}
\inputencoding{utf8}
\HeaderA{zipfpeVariance}{Variance of the Zipf-PE distribution.}{zipfpeVariance}
%
\begin{Description}\relax
Computes the variance of the Zipf-PE distribution for given values of \eqn{\alpha}{} and \eqn{\beta}{}.
\end{Description}
%
\begin{Usage}
\begin{verbatim}
zipfpeVariance(alpha, beta, tolerance = 10^(-4))
\end{verbatim}
\end{Usage}
%
\begin{Arguments}
\begin{ldescription}
\item[\code{alpha}] Value of the \eqn{\alpha}{} parameter (\eqn{\alpha > 3}{}).

\item[\code{beta}] Value of the \eqn{\beta}{} parameter (\eqn{\beta \in (-\infty, +\infty)}{}).

\item[\code{tolerance}] Tolerance used in the calculations. (default = \eqn{10^{-4}}{})
\end{ldescription}
\end{Arguments}
%
\begin{Details}\relax
The variance of the distribution only exists for \eqn{\alpha}{} strictly greater than 3.
In this case, it is calculated as:
\deqn{Var[Y] = E[Y^2] - (E[Y])^2}{},
where the first and the second moments are computed using the \emph{zipfpeMoments} function.
\end{Details}
%
\begin{Value}
A positive real value corresponding to the variance of the distribution.
\end{Value}
%
\begin{Examples}
\begin{ExampleCode}
zipfpeVariance(3.5, 1.3)
\end{ExampleCode}
\end{Examples}
\inputencoding{utf8}
\HeaderA{zipfpss}{The Zipf-Poisson Stop Sum Distribution (Zipf-PSS).}{zipfpss}
\aliasA{dzipfpss}{zipfpss}{dzipfpss}
\aliasA{dzipfpss}{zipfpss}{dzipfpss}
\aliasA{pzipfpss}{zipfpss}{pzipfpss}
\aliasA{qzipfpss}{zipfpss}{qzipfpss}
\aliasA{rzipfpss}{zipfpss}{rzipfpss}
%
\begin{Description}\relax
Probability mass function, cumulative function, quantile function and random number generation
for the Zipf-PSS distribution with parameters \eqn{\alpha}{} and \eqn{\lambda}{}.
\end{Description}
%
\begin{Usage}
\begin{verbatim}
dzipfpss(x, alpha, lambda, log = FALSE, isTruncated = FALSE)

pzipfpss(q, alpha, lambda, log.p = FALSE, lower.tail = TRUE,
  isTruncated = FALSE)

rzipfpss(n, alpha, lambda, log.p = FALSE, lower.tail = TRUE,
  isTruncated = FALSE)

qzipfpss(p, alpha, lambda, log.p = FALSE, lower.tail = TRUE,
  isTruncated = FALSE)
\end{verbatim}
\end{Usage}
%
\begin{Arguments}
\begin{ldescription}
\item[\code{x, q}] Vector of positive integer values.

\item[\code{alpha}] Value of the \eqn{\alpha}{} parameter (\eqn{\alpha > 1}{} ).

\item[\code{lambda}] Value of the \eqn{\lambda}{} parameter (\eqn{\lambda \geq 0}{} ).

\item[\code{log, log.p}] Logical; if TRUE, probabilities p are given as log(p).

\item[\code{isTruncated}] Logical; if TRUE, the truncated version of the distribution is returned.

\item[\code{lower.tail}] Logical; if TRUE (default), probabilities are \eqn{P[X \leq x]}{}, otherwise, \eqn{P[X > x]}{}.

\item[\code{n}] Number of random values to return.

\item[\code{p}] Vector of probabilities.
\end{ldescription}
\end{Arguments}
%
\begin{References}\relax

Bjørn Sundt and William S Jewell. 1981. Further results on recursive evaluation of compound distributions. ASTIN
Bulletin: The Journal of the IAA 12, 1 (1981), 27–39.

Harry H Panjer. 1981. Recursive evaluation of a family of compound distributions. Astin Bulletin 12, 01 (1981), 22–26.

\end{References}
\inputencoding{utf8}
\HeaderA{zipfpssFit}{Zipf-PSS parameters estimation.}{zipfpssFit}
\aliasA{AIC.zipfpssR}{zipfpssFit}{AIC.zipfpssR}
\aliasA{BIC.zipfpssR}{zipfpssFit}{BIC.zipfpssR}
\aliasA{coef.zipfpssR}{zipfpssFit}{coef.zipfpssR}
\aliasA{fitted.zipfpssR}{zipfpssFit}{fitted.zipfpssR}
\aliasA{logLik.zipfpssR}{zipfpssFit}{logLik.zipfpssR}
\aliasA{plot.zipfpssR}{zipfpssFit}{plot.zipfpssR}
\aliasA{print.zipfpssR}{zipfpssFit}{print.zipfpssR}
\aliasA{residuals.zipfpssR}{zipfpssFit}{residuals.zipfpssR}
\aliasA{summary.zipfpssR}{zipfpssFit}{summary.zipfpssR}
%
\begin{Description}\relax
For a given sample of strictly positive integer numbers,  usually of the type of ranking data or
frequencies of frequencies data, estimates the parameters of a Zipf-PSS distribution by means of
the maximum likelihood method. Note that the input data should be provided as a frequency matrix.
\end{Description}
%
\begin{Usage}
\begin{verbatim}
zipfpssFit(data, init_alpha, init_lambda, level = 0.95, isTruncated = FALSE,
  ...)

## S3 method for class 'zipfpssR'
residuals(object, ...)

## S3 method for class 'zipfpssR'
fitted(object, ...)

## S3 method for class 'zipfpssR'
coef(object, ...)

## S3 method for class 'zipfpssR'
plot(x, ...)

## S3 method for class 'zipfpssR'
print(x, ...)

## S3 method for class 'zipfpssR'
summary(object, ...)

## S3 method for class 'zipfpssR'
logLik(object, ...)

## S3 method for class 'zipfpssR'
AIC(object, ...)

## S3 method for class 'zipfpssR'
BIC(object, ...)
\end{verbatim}
\end{Usage}
%
\begin{Arguments}
\begin{ldescription}
\item[\code{data}] Matrix of count data in form of table of frequencies.

\item[\code{init\_alpha}] Initial value of \eqn{\alpha}{} parameter (\eqn{\alpha > 1}{}).

\item[\code{init\_lambda}] Initial value of \eqn{\lambda}{} parameter (\eqn{\lambda \geq 0}{}).

\item[\code{level}] Confidence level used to calculate the confidence intervals (default 0.95).

\item[\code{isTruncated}] Logical; if TRUE, the truncated version of the distribution is returned.(default = FALSE)

\item[\code{...}] Further arguments to the generic functions. The extra arguments are passing
to the \emph{\LinkA{optim}{optim}} function.

\item[\code{object}] An object from class "zpssR" (output of \emph{zipfpssFit} function).

\item[\code{x}] An object from class "zpssR" (output of \emph{zipfpssFit} function).
\end{ldescription}
\end{Arguments}
%
\begin{Details}\relax
The argument \code{data} is a two column matrix such that the first column of each row contains a
count, while the corresponding second column contains its frequency.

The log-likelihood function is equalt to:
\deqn{l(\alpha, \lambda, x) = \sum_{i =1} ^{m} f_a(x_i)\, log(P(Y = x_i))}{},
where \eqn{m}{} is the number of different values in the sample, being \eqn{f_{a}(x_i)}{} is the absolute
frequency of \eqn{x_i}{}.The probabilities are calculated applying the Panjer recursion.

The function \emph{\LinkA{optim}{optim}} is used to estimate the parameters.
\end{Details}
%
\begin{Value}
Returns a \emph{zpssR} object composed by the maximum likelihood parameter estimations,
their standard deviation, their confidence intervals and the value of the log-likelihood at the
maximum likelihood estimator.
\end{Value}
%
\begin{References}\relax

Bjørn Sundt and William S Jewell. 1981. Further results on recursive evaluation of compound distributions. ASTIN
Bulletin: The Journal of the IAA 12, 1 (1981), 27–39.

Harry H Panjer. 1981. Recursive evaluation of a family of compound distributions. Astin Bulletin 12, 01 (1981), 22–26.

\end{References}
%
\begin{SeeAlso}\relax
\code{\LinkA{moezipf\_getInitialValues}{moezipf.Rul.getInitialValues}}.
\end{SeeAlso}
%
\begin{Examples}
\begin{ExampleCode}
data <- rzipfpss(100, 2.5, 1.3)
data <- as.data.frame(table(data))
data[,1] <- as.numeric(data[,1])
obj <- zipfpssFit(data, 1.1, 0.1)
\end{ExampleCode}
\end{Examples}
\inputencoding{utf8}
\HeaderA{zipfpssMean}{Expected value of the Zipf-PSS distribution.}{zipfpssMean}
%
\begin{Description}\relax
Computes the expected value of the Zipf-PSS distribution for given values of parameters
\eqn{\alpha}{} and \eqn{\lambda}{}.
\end{Description}
%
\begin{Usage}
\begin{verbatim}
zipfpssMean(alpha, lambda, isTruncated = FALSE)
\end{verbatim}
\end{Usage}
%
\begin{Arguments}
\begin{ldescription}
\item[\code{alpha}] Value of the \eqn{\alpha}{} parameter (\eqn{\alpha > 2}{}).

\item[\code{lambda}] Value of the \eqn{\lambda}{} parameter (\eqn{\lambda \geq 0}{}).

\item[\code{isTruncated}] Logical; if TRUE Use the zero-truncated version of the distribution to calculate the expected value (default = FALSE).
\end{ldescription}
\end{Arguments}
%
\begin{Details}\relax
The expected value of the Zipf-PSS distribution only exists for \eqn{\alpha}{} values strictly
greater than 2. The value is derived from \eqn{E[Y] = E[N]\, E[X]}{} where E[X] is the mean value
of the Zipf distribution and E[N] is the expected value of a Poisson one.

The resulting expression is set to be equal to:
\deqn{E[Y] = \lambda\, \frac{\zeta(\alpha - 1)}{\zeta(\alpha)}}{}

Particularlly, if one is dealing with the zero-truncated version of the Zipf-PSS distribution.
This values is calculated as:
\deqn{E[Y^{ZT}] = \frac{\lambda\, \zeta(\alpha - 1)}{\zeta(\alpha)\, (1 - e^{-\lambda})}}{}
\end{Details}
%
\begin{Value}
A positive real value corresponding to the mean value of the distribution.
\end{Value}
%
\begin{References}\relax

Sarabia Alegría, JM. and Gómez Déniz, E. and Vázquez Polo, F. Estadística actuarial: teoría y aplicaciones. Pearson Prentice Hall.

\end{References}
%
\begin{Examples}
\begin{ExampleCode}
zipfpssMean(2.5, 1.3)
zipfpssMean(2.5, 1.3, TRUE)
\end{ExampleCode}
\end{Examples}
\inputencoding{utf8}
\HeaderA{zipfpssMoments}{Distribution Moments.}{zipfpssMoments}
%
\begin{Description}\relax
General function to compute the k-th moment of the Zipf-PSS distribution, for any integer value \eqn{k \geq 1}{}
when it exists. Note that the k-th moment exists if and only if  \eqn{\alpha > k + 1}{}.
\end{Description}
%
\begin{Usage}
\begin{verbatim}
zipfpssMoments(k, alpha, lambda, isTruncated = FALSE, tolerance = 10^(-4))
\end{verbatim}
\end{Usage}
%
\begin{Arguments}
\begin{ldescription}
\item[\code{k}] Order of the moment to compute.

\item[\code{alpha}] Value of the \eqn{\alpha}{} parameter (\eqn{\alpha > k + 1}{}).

\item[\code{lambda}] Value of the \eqn{\lambda}{} parameter (\eqn{\lambda \geq 0}{}).

\item[\code{isTruncated}] Logical; if TRUE, the truncated version of the distribution is returned.

\item[\code{tolerance}] Tolerance used in the calculations (default = \eqn{10^{-4}}{}).
\end{ldescription}
\end{Arguments}
%
\begin{Details}\relax
The k-th moment of the Zipf-PSS distribution is finite for \eqn{\alpha}{} values
strictly greater than \eqn{k + 1}{}.
It is computed calculating the partial sums of the serie, and stopping when two
consecutive partial sums differs less than the \code{tolerance} value.
The value of the last partial sum is returned.
\end{Details}
%
\begin{Value}
A positive real value corresponding to the k-th moment of the distribution.
\end{Value}
%
\begin{Examples}
\begin{ExampleCode}
zipfpssMoments(1, 2.5, 2.3)
zipfpssMoments(1, 2.5, 2.3, TRUE)
\end{ExampleCode}
\end{Examples}
\inputencoding{utf8}
\HeaderA{zipfpssVariance}{Variance of the Zipf-PSS distribution.}{zipfpssVariance}
%
\begin{Description}\relax
Computes the variance of the Zipf-PSS distribution for given values of parameters
\eqn{\alpha}{} and \eqn{\lambda}{}.
\end{Description}
%
\begin{Usage}
\begin{verbatim}
zipfpssVariance(alpha, lambda, isTruncated = FALSE)
\end{verbatim}
\end{Usage}
%
\begin{Arguments}
\begin{ldescription}
\item[\code{alpha}] Value of the \eqn{\alpha}{} parameter (\eqn{\alpha > 3}{}).

\item[\code{lambda}] Value of the \eqn{\lambda}{} parameter (\eqn{\lambda \geq 0}{}).

\item[\code{isTruncated}] Logical; if TRUE Use the zero-truncated version of the distribution to calculate the expected value (default = FALSE).
\end{ldescription}
\end{Arguments}
%
\begin{Details}\relax
The variance of the Zipf-PSS distribution only exists for \eqn{\alpha}{} values strictly greater than 3.
The value is derived from \eqn{Var[Y] = E[N]\, Var[X] + E[X]^2 \, Var[N]}{} where E[X] and
E[N] are the expected value of the Zipf and the Poisson distributions respectively.
In the same way the values of Var[X] and Var[N] stand for the variances of the Zipf and
the Poisson distributions. The resulting expression is set to be equal to:

\deqn{Var[Y] = \lambda\, \frac{\zeta(\alpha - 2)}{\zeta(\alpha)}}{}
Particularlly, the variance of the zero-truncated version of the Zipf-PSS distribution is
calculated as:
\deqn{Var[Y^{ZT}] = \frac{\lambda\, \zeta(\alpha)\, \zeta(\alpha - 2)\, (1 - e^{-\lambda}) - \lambda^2 \, \zeta(\alpha - 1)^2 \, e^{-\lambda}}{\zeta(\alpha)^2 \, (1 - e^{-\lambda})^2}}{}
\end{Details}
%
\begin{Value}
A positive real value corresponding to the variance of the distribution.
\end{Value}
%
\begin{References}\relax

Sarabia Alegría, JM. and Gómez Déniz, E. and Vázquez Polo, F. Estadística actuarial: teoría y aplicaciones. Pearson Prentice Hall.

\end{References}
%
\begin{Examples}
\begin{ExampleCode}
zipfpssVariance(4.5, 2.3)
zipfpssVariance(4.5, 2.3, TRUE)
\end{ExampleCode}
\end{Examples}
\printindex{}
\end{document}
